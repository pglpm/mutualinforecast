\pdfoutput=1
%% Author: PGL  Porta Mana, C Battistin, S Gonzalo Cogno
%% Created: 2015-05-01T20:53:34+0200
%% Last-Updated: 2021-03-15T10:46:39+0100
%%%%%%%%%%%%%%%%%%%%%%%%%%%%%%%%%%%%%%%%%%%%%%%%%%%%%%%%%%%%%%%%%%%%%%%%%%%%
\newif\ifarxiv
\arxivfalse
\ifarxiv\pdfmapfile{+classico.map}\fi
\newif\ifafour
\afourfalse% true = A4, false = A5
\newif\iftypodisclaim % typographical disclaim on the side
\typodisclaimtrue
\newcommand*{\memfontfamily}{zplx}
\newcommand*{\memfontpack}{newpxtext}
\documentclass[\ifafour a4paper,12pt,\else a5paper,10pt,\fi%extrafontsizes,%
onecolumn,oneside,article,%french,italian,german,swedish,latin,
british%
]{memoir}
\newcommand*{\firstdraft}{31 March 2019}
\newcommand*{\firstpublished}{\firstdraft}
\newcommand*{\updated}{\ifarxiv***\else\today\fi}
\newcommand*{\propertitle}{Reasoned inference of long-run mutual
  information\\\normalsize(Bayesian theory for dummies) [draft]%\\{\large ***}%
}% title uses LARGE; set Large for smaller
\newcommand*{\pdftitle}{Reasoned inference of long-run mutual information}
\newcommand*{\headtitle}{Guessing long-run mutual information}
\newcommand*{\pdfauthor}{P.G.L.  Porta Mana}
\newcommand*{\headauthor}{Porta Mana, Battistin, Gonzalo Cogno}
\newcommand*{\reporthead}{\ifarxiv\else Open Science Framework \href{https://doi.org/10.31219/osf.io/***}{\textsc{doi}:10.31219/osf.io/***}\fi}% Report number

%%%%%%%%%%%%%%%%%%%%%%%%%%%%%%%%%%%%%%%%%%%%%%%%%%%%%%%%%%%%%%%%%%%%%%%%%%%%
%%% Calls to packages (uncomment as needed)
%%%%%%%%%%%%%%%%%%%%%%%%%%%%%%%%%%%%%%%%%%%%%%%%%%%%%%%%%%%%%%%%%%%%%%%%%%%%

\usepackage[T1]{fontenc} 
\input{glyphtounicode} \pdfgentounicode=1

\usepackage[utf8]{inputenx}

\newcommand*{\bmmax}{0} % reduce number of bold fonts, before font packages
\newcommand*{\hmmax}{0} % reduce number of heavy fonts, before font packages

\usepackage{textcomp}

\usepackage{amsmath}

\usepackage{mathtools}
%\addtolength{\jot}{\jot} % increase spacing in multiline formulae
\setlength{\multlinegap}{0pt}

%\usepackage{empheq}% automatically calls amsmath and mathtools
%\newcommand*{\widefbox}[1]{\fbox{\hspace{1em}#1\hspace{1em}}}

\usepackage{amssymb}

\usepackage{amsxtra}

\usepackage[main=british]{babel}\selectlanguage{british}
\newcommand{\langnohyph}[1]{\begin{hyphenrules}{nohyphenation}#1\end{hyphenrules}}

\usepackage[autostyle=false,autopunct=false,english=british]{csquotes}
\setquotestyle{american}
\newcommand*{\defquote}[1]{`\,#1\,'}

% \makeatletter
% \renewenvironment{quotation}%
%                {\list{}{\listparindent 1.5em%
%                         \itemindent    \listparindent
%                         \rightmargin=1em   \leftmargin=1em
%                         \parsep        \z@ \@plus\p@}%
%                 \item[]\footnotesize}%
%                 {\endlist}
% \makeatother                


\usepackage{amsthm}
\newcommand*{\QED}{\textsc{q.e.d.}}
\renewcommand*{\qedsymbol}{\QED}
\theoremstyle{remark}
\newtheorem{note}{Note}
\newtheorem*{remark}{Note}
\newtheoremstyle{innote}{\parsep}{\parsep}{\footnotesize}{}{}{}{0pt}{}
\theoremstyle{innote}
\newtheorem*{innote}{}

\usepackage[shortlabels,inline]{enumitem}
\SetEnumitemKey{para}{itemindent=\parindent,leftmargin=0pt,listparindent=\parindent,parsep=0pt,itemsep=\topsep}
\setlist{itemsep=0pt,topsep=\parsep}
\setlist[enumerate,2]{label=\alph*.}
\setlist[enumerate]{label=\arabic*.,leftmargin=1.5\parindent}
\setlist[itemize]{leftmargin=1.5\parindent}
\setlist[description]{leftmargin=1.5\parindent}

\usepackage[babel,theoremfont,largesc]{newpxtext}

\usepackage[bigdelims,nosymbolsc%,smallerops % probably arXiv doesn't have it
]{newpxmath}
\linespread{1.1}% best for text with maths
%% smaller operators for old version of newpxmath
\makeatletter
\def\re@DeclareMathSymbol#1#2#3#4{%
    \let#1=\undefined
    \DeclareMathSymbol{#1}{#2}{#3}{#4}}
%\re@DeclareMathSymbol{\bigsqcupop}{\mathop}{largesymbols}{"46}
%\re@DeclareMathSymbol{\bigodotop}{\mathop}{largesymbols}{"4A}
\re@DeclareMathSymbol{\bigoplusop}{\mathop}{largesymbols}{"4C}
\re@DeclareMathSymbol{\bigotimesop}{\mathop}{largesymbols}{"4E}
\re@DeclareMathSymbol{\sumop}{\mathop}{largesymbols}{"50}
\re@DeclareMathSymbol{\prodop}{\mathop}{largesymbols}{"51}
\re@DeclareMathSymbol{\bigcupop}{\mathop}{largesymbols}{"53}
\re@DeclareMathSymbol{\bigcapop}{\mathop}{largesymbols}{"54}
%\re@DeclareMathSymbol{\biguplusop}{\mathop}{largesymbols}{"55}
\re@DeclareMathSymbol{\bigwedgeop}{\mathop}{largesymbols}{"56}
\re@DeclareMathSymbol{\bigveeop}{\mathop}{largesymbols}{"57}
%\re@DeclareMathSymbol{\bigcupdotop}{\mathop}{largesymbols}{"DF}
%\re@DeclareMathSymbol{\bigcapplusop}{\mathop}{largesymbolsPXA}{"00}
%\re@DeclareMathSymbol{\bigsqcupplusop}{\mathop}{largesymbolsPXA}{"02}
%\re@DeclareMathSymbol{\bigsqcapplusop}{\mathop}{largesymbolsPXA}{"04}
%\re@DeclareMathSymbol{\bigsqcapop}{\mathop}{largesymbolsPXA}{"06}
\re@DeclareMathSymbol{\bigtimesop}{\mathop}{largesymbolsPXA}{"10}
%\re@DeclareMathSymbol{\coprodop}{\mathop}{largesymbols}{"60}
%\re@DeclareMathSymbol{\varprod}{\mathop}{largesymbolsPXA}{16}
\makeatother
%%
%% With euler font cursive for Greek letters - the [1] means 100% scaling
\DeclareFontFamily{U}{egreek}{\skewchar\font'177}%
\DeclareFontShape{U}{egreek}{m}{n}{<-6>s*[1]eurm5 <6-8>s*[1]eurm7 <8->s*[1]eurm10}{}%
\DeclareFontShape{U}{egreek}{m}{it}{<->s*[1]eurmo10}{}%
\DeclareFontShape{U}{egreek}{b}{n}{<-6>s*[1]eurb5 <6-8>s*[1]eurb7 <8->s*[1]eurb10}{}%
\DeclareFontShape{U}{egreek}{b}{it}{<->s*[1]eurbo10}{}%
\DeclareSymbolFont{egreeki}{U}{egreek}{m}{it}%
\SetSymbolFont{egreeki}{bold}{U}{egreek}{b}{it}% from the amsfonts package
\DeclareSymbolFont{egreekr}{U}{egreek}{m}{n}%
\SetSymbolFont{egreekr}{bold}{U}{egreek}{b}{n}% from the amsfonts package
% Take also \sum, \prod, \coprod symbols from Euler fonts
\DeclareFontFamily{U}{egreekx}{\skewchar\font'177}
\DeclareFontShape{U}{egreekx}{m}{n}{%
       <-7.5>s*[0.9]euex7%
    <7.5-8.5>s*[0.9]euex8%
    <8.5-9.5>s*[0.9]euex9%
    <9.5->s*[0.9]euex10%
}{}
\DeclareSymbolFont{egreekx}{U}{egreekx}{m}{n}
\DeclareMathSymbol{\sumop}{\mathop}{egreekx}{"50}
\DeclareMathSymbol{\prodop}{\mathop}{egreekx}{"51}
\DeclareMathSymbol{\coprodop}{\mathop}{egreekx}{"60}
\makeatletter
\def\sum{\DOTSI\sumop\slimits@}
\def\prod{\DOTSI\prodop\slimits@}
\def\coprod{\DOTSI\coprodop\slimits@}
\makeatother
\input{definegreek.tex}% Greek letters not usually given in LaTeX.

\renewcommand\sfdefault{uop}
\DeclareMathAlphabet{\mathsf}  {T1}{\sfdefault}{m}{sl}
\SetMathAlphabet{\mathsf}{bold}{T1}{\sfdefault}{b}{sl}

\usepackage[scaled=0.84]{DejaVuSansMono}% DejaVu Mono as typewriter text


\usepackage{mathdots}

\usepackage[usenames]{xcolor}
% Tol (2012) colour-blind-, print-, screen-friendly colours, alternative scheme; Munsell terminology
\definecolor{mypurpleblue}{RGB}{68,119,170}
\definecolor{myblue}{RGB}{102,204,238}
\definecolor{mygreen}{RGB}{34,136,51}
\definecolor{myyellow}{RGB}{204,187,68}
\definecolor{myred}{RGB}{238,102,119}
\definecolor{myredpurple}{RGB}{170,51,119}
\definecolor{mygrey}{RGB}{187,187,187}
% Tol (2012) colour-blind-, print-, screen-friendly colours; Munsell terminology
% \definecolor{lbpurple}{RGB}{51,34,136}
% \definecolor{lblue}{RGB}{136,204,238}
% \definecolor{lbgreen}{RGB}{68,170,153}
% \definecolor{lgreen}{RGB}{17,119,51}
% \definecolor{lgyellow}{RGB}{153,153,51}
% \definecolor{lyellow}{RGB}{221,204,119}
% \definecolor{lred}{RGB}{204,102,119}
% \definecolor{lpred}{RGB}{136,34,85}
% \definecolor{lrpurple}{RGB}{170,68,153}
\definecolor{lgrey}{RGB}{221,221,221}
\colorlet{shadecolor}{lgrey}

\usepackage{bm}

\usepackage{microtype}

\usepackage[backend=biber,mcite,%subentry,
citestyle=authoryear-comp,bibstyle=pglpm-authoryear,autopunct=false,sorting=ny,sortcites=false,natbib=false,maxcitenames=2,maxbibnames=8,minbibnames=8,giveninits=true,uniquename=false,uniquelist=false,maxalphanames=1,block=space,hyperref=true,defernumbers=false,useprefix=true,sortupper=false,language=british,parentracker=false]{biblatex}
\DeclareSortingTemplate{ny}{\sort{\field{sortname}\field{author}\field{editor}}\sort{\field{year}}}
\iffalse\makeatletter%%% replace parenthesis with brackets
\newrobustcmd*{\parentexttrack}[1]{%
  \begingroup
  \blx@blxinit
  \blx@setsfcodes
  \blx@bibopenparen#1\blx@bibcloseparen
  \endgroup}
\AtEveryCite{%
  \let\parentext=\parentexttrack%
  \let\bibopenparen=\bibopenbracket%
  \let\bibcloseparen=\bibclosebracket}
\makeatother\fi
\DefineBibliographyExtras{british}{\def\finalandcomma{\addcomma}}
\renewcommand*{\finalnamedelim}{\addspace\amp\space}
%\renewcommand*{\finalnamedelim}{\addcomma\space}
\setcounter{biburlnumpenalty}{1}
\setcounter{biburlucpenalty}{0}
\setcounter{biburllcpenalty}{1}
\DeclareDelimFormat{multicitedelim}{\addsemicolon\addspace\space}
\DeclareDelimFormat{compcitedelim}{\addsemicolon\addspace\space}
\DeclareDelimFormat{postnotedelim}{\addspace}
\ifarxiv\else\addbibresource{portamanabib.bib}\fi
\renewcommand{\bibfont}{\footnotesize}
\defbibheading{bibliography}[\bibname]{\section*{#1}\addcontentsline{toc}{section}{#1}%\markboth{#1}{#1}
}
\newcommand*{\citep}{\footcites}
\newcommand*{\citey}{\footcites}%{\parencites*}
\newcommand*{\ibid}{\unspace\addtocounter{footnote}{-1}\footnotemark{}}
\providecommand{\href}[2]{#2}
\providecommand{\eprint}[2]{\texttt{\href{#1}{#2}}}
\newcommand*{\amp}{\&}
\newcommand*{\citein}[2][]{\textnormal{\textcite[#1]{#2}}}
\newcommand*{\citebi}[2][]{\textcite[#1]{#2}}
\newcommand*{\subtitleproc}[1]{}
\newcommand*{\chapb}{ch.}

\newcommand*{\arxiveprint}[1]{\texttt{arXiv:\urlalt{https://arxiv.org/abs/#1}{#1}}}
\newcommand*{\mparceprint}[1]{\texttt{mp\_arc:\urlalt{http://www.ma.utexas.edu/mp_arc-bin/mpa?yn=#1}{#1}}}
\newcommand*{\haleprint}[1]{\texttt{HAL:\urlalt{https://hal.archives-ouvertes.fr/#1}{#1}}}
\newcommand*{\philscieprint}[1]{\texttt{PhilSci:\urlalt{http://philsci-archive.pitt.edu/archive/#1}{#1}}}
\newcommand*{\biorxiveprint}[1]{bioRxiv \texttt{doi:\urlalt{https://doi.org/10.1101/#1}{10.1101/#1}}}
\newcommand*{\osfeprint}[1]{Open Science Framework \texttt{doi:\urlalt{https://doi.org/10.17605/osf.io/#1}{10.17605/osf.io/#1}}}

\usepackage{graphicx}

\usepackage{wrapfig}

\PassOptionsToPackage{hyphens}{url}\usepackage[hypertexnames=false]{hyperref}

\usepackage[depth=4]{bookmark}
\hypersetup{colorlinks=true,bookmarksnumbered,pdfborder={0 0 0.25},citebordercolor={0.2667 0.4667 0.6667},citecolor=mypurpleblue,linkbordercolor={0.6667 0.2 0.4667},linkcolor=myredpurple,urlbordercolor={0.1333 0.5333 0.2},urlcolor=mygreen,breaklinks=true,pdftitle={\pdftitle},pdfauthor={\pdfauthor}}
% \usepackage[vertfit=local]{breakurl}% only for arXiv
\providecommand*{\urlalt}{\href}

\usepackage[british]{datetime2}
\DTMnewdatestyle{mydate}%
{% definitions
\renewcommand*{\DTMdisplaydate}[4]{%
\number##3\ \DTMenglishmonthname{##2} ##1}%
\renewcommand*{\DTMDisplaydate}{\DTMdisplaydate}%
}
\DTMsetdatestyle{mydate}

%%%%%%%%%%%%%%%%%%%%%%%%%%%%%%%%%%%%%%%%%%%%%%%%%%%%%%%%%%%%%%%%%%%%%%%%%%%%
%%% Layout. I do not know on which kind of paper the reader will print the
%%% paper on (A4? letter? one-sided? double-sided?). So I choose A5, which
%%% provides a good layout for reading on screen and save paper if printed
%%% two pages per sheet. Average length line is 66 characters and page
%%% numbers are centred.
%%%%%%%%%%%%%%%%%%%%%%%%%%%%%%%%%%%%%%%%%%%%%%%%%%%%%%%%%%%%%%%%%%%%%%%%%%%%
\ifafour\setstocksize{297mm}{210mm}%{*}% A4
\else\setstocksize{210mm}{5.5in}%{*}% 210x139.7
\fi
\settrimmedsize{\stockheight}{\stockwidth}{*}
\setlxvchars[\normalfont] %313.3632pt for a 66-characters line
\setxlvchars[\normalfont]
\setlength{\trimtop}{0pt}
\setlength{\trimedge}{\stockwidth}
\addtolength{\trimedge}{-\paperwidth}
% The length of the normalsize alphabet is 133.05988pt - 10 pt = 26.1408pc
% The length of the normalsize alphabet is 159.6719pt - 12pt = 30.3586pc
% Bringhurst gives 32pc as boundary optimal with 69 ch per line
% The length of the normalsize alphabet is 191.60612pt - 14pt = 35.8634pc
\ifafour\settypeblocksize{*}{32pc}{1.618} % A4
%\setulmargins{*}{*}{1.667}%gives 5/3 margins % 2 or 1.667
\else\settypeblocksize{*}{26pc}{1.618}% nearer to a 66-line newpx and preserves GR
\fi
\setulmargins{*}{*}{1}%gives equal margins
\setlrmargins{*}{*}{*}
\setheadfoot{\onelineskip}{2.5\onelineskip}
\setheaderspaces{*}{2\onelineskip}{*}
\setmarginnotes{2ex}{10mm}{0pt}
\checkandfixthelayout[nearest]
%%% End layout
%% this fixes missing white spaces
\pdfinterwordspaceon%

%%% Sectioning
\newcommand*{\asudedication}[1]{%
{\par\centering\textit{#1}\par}}
\newenvironment{acknowledgements}{\section*{Thanks}\addcontentsline{toc}{section}{Thanks}}{\par}
\makeatletter\renewcommand{\appendix}{\par
  \bigskip{\centering
   \interlinepenalty \@M
   \normalfont
   \printchaptertitle{\sffamily\appendixpagename}\par}
  \setcounter{section}{0}%
  \gdef\@chapapp{\appendixname}%
  \gdef\thesection{\@Alph\c@section}%
  \anappendixtrue}\makeatother
\counterwithout{section}{chapter}
\setsecnumformat{\upshape\csname the#1\endcsname\quad}
\setsecheadstyle{\large\bfseries\sffamily%
\centering}
\setsubsecheadstyle{\bfseries\sffamily%
\raggedright}
\setsubsecindent{0pt}%0ex plus 1ex minus .2ex}
\setparaheadstyle{\bfseries\sffamily%
\raggedright}
\setcounter{secnumdepth}{2}
\setlength{\headwidth}{\textwidth}
\newcommand{\addchap}[1]{\chapter*[#1]{#1}\addcontentsline{toc}{chapter}{#1}}
\newcommand{\addsec}[1]{\section*{#1}\addcontentsline{toc}{section}{#1}}
\newcommand{\addsubsec}[1]{\subsection*{#1}\addcontentsline{toc}{subsection}{#1}}
\newcommand{\addpara}[1]{\paragraph*{#1.}\addcontentsline{toc}{subsubsection}{#1}}
\newcommand{\addparap}[1]{\paragraph*{#1}\addcontentsline{toc}{subsubsection}{#1}}

%%% Headers, footers, pagestyle
\copypagestyle{manaart}{plain}
\makeheadrule{manaart}{\headwidth}{0.5\normalrulethickness}
\makeoddhead{manaart}{%
{\footnotesize%\sffamily%
\scshape\headauthor}}{}{{\footnotesize\sffamily%
\headtitle}}
\makeoddfoot{manaart}{}{\thepage}{}
\newcommand*\autanet{\includegraphics[height=\heightof{M}]{autanet.pdf}}
\definecolor{mygray}{gray}{0.333}
\iftypodisclaim%
\ifafour\newcommand\addprintnote{\begin{picture}(0,0)%
\put(245,149){\makebox(0,0){\rotatebox{90}{\tiny\color{mygray}\textsf{This
            document is designed for screen reading and
            two-up printing on A4 or Letter paper}}}}%
\end{picture}}% A4
\else\newcommand\addprintnote{\begin{picture}(0,0)%
\put(176,112){\makebox(0,0){\rotatebox{90}{\tiny\color{mygray}\textsf{This
            document is designed for screen reading and
            two-up printing on A4 or Letter paper}}}}%
\end{picture}}\fi%afourtrue
\makeoddfoot{plain}{}{\makebox[0pt]{\thepage}\addprintnote}{}
\else
\makeoddfoot{plain}{}{\makebox[0pt]{\thepage}}{}
\fi%typodisclaimtrue
\makeoddhead{plain}{\scriptsize\reporthead}{}{}

\pagestyle{manaart}

\setlength{\droptitle}{-3.9\onelineskip}
\pretitle{\begin{center}\Large\sffamily%
\bfseries}
\posttitle{\bigskip\end{center}}

\makeatletter\newcommand*{\atf}{\includegraphics[totalheight=\heightof{@}]{atblack.png}}\makeatother
\providecommand{\affiliation}[1]{\textsl{\textsf{\footnotesize #1}}}
\providecommand{\epost}[1]{\texttt{\footnotesize\textless#1\textgreater}}
\providecommand{\email}[2]{\href{mailto:#1ZZ@#2 ((remove ZZ))}{#1\protect\atf#2}}

\preauthor{\vspace{-0.5\baselineskip}\begin{center}
\normalsize\sffamily%
\lineskip  0.5em}
\postauthor{\par\end{center}}
\predate{\DTMsetdatestyle{mydate}\begin{center}\footnotesize}
\postdate{\end{center}\vspace{-\medskipamount}}

\setfloatadjustment{figure}{\footnotesize}
\captiondelim{\quad}
\captionnamefont{\footnotesize\sffamily%
}
\captiontitlefont{\footnotesize}
\midsloppy
% handling orphan/widow lines, memman.pdf
% \clubpenalty=10000
% \widowpenalty=10000
% \raggedbottom
% Downes, memman.pdf
\clubpenalty=9996
\widowpenalty=9999
\brokenpenalty=4991
\predisplaypenalty=10000
\postdisplaypenalty=1549
\displaywidowpenalty=1602
\raggedbottom

\paragraphfootnotes
\setlength{\footmarkwidth}{2ex}
\footmarkstyle{\textsuperscript{%\color{myred}
\scriptsize\bfseries#1}~}

\selectlanguage{british}\frenchspacing

%%%%%%%%%%%%%%%%%%%%%%%%%%%%%%%%%%%%%%%%%%%%%%%%%%%%%%%%%%%%%%%%%%%%%%%%%%%%
%%% Paper's details
%%%%%%%%%%%%%%%%%%%%%%%%%%%%%%%%%%%%%%%%%%%%%%%%%%%%%%%%%%%%%%%%%%%%%%%%%%%%
\title{\propertitle}
\author{%
\hspace*{\stretch{1}}%
\parbox{0.30\linewidth}%\makebox[0pt][c]%
{\protect\centering P.G.L.  Porta Mana  \href{https://orcid.org/0000-0002-6070-0784}{\protect\includegraphics[scale=0.16]{orcid_32x32.png}}\\%
\footnotesize\epost{\email{pgl}{portamana.org}}}%
\hspace*{\stretch{1}}%
\parbox{0.35\linewidth}%\makebox[0pt][c]%
{\protect\centering C. Battistin\\%
\footnotesize\epost{\email{claudia.battistin}{ntnu.no}}}%
\hspace*{\stretch{1}}%
\parbox{0.33\linewidth}%\makebox[0pt][c]%
{\protect\centering S. Gonzalo Cogno\\%
\footnotesize\epost{\email{soledad.g.cogno}{ntnu.no}}}%
\hspace*{\stretch{1}}%
\\
\footnotesize(or any permutation thereof)}

\date{\firstdraft; updated \updated}

%%%%%%%%%%%%%%%%%%%%%%%%%%%%%%%%%%%%%%%%%%%%%%%%%%%%%%%%%%%%%%%%%%%%%%%%%%%%
%%% Macros @@@
%%%%%%%%%%%%%%%%%%%%%%%%%%%%%%%%%%%%%%%%%%%%%%%%%%%%%%%%%%%%%%%%%%%%%%%%%%%%

% Common ones - uncomment as needed
%\providecommand{\nequiv}{\not\equiv}
%\providecommand{\coloneqq}{\mathrel{\mathop:}=}
%\providecommand{\eqqcolon}{=\mathrel{\mathop:}}
%\providecommand{\varprod}{\prod}
\newcommand*{\de}{\partialup}%partial diff
\newcommand*{\pu}{\piup}%constant pi
\newcommand*{\delt}{\deltaup}%Kronecker, Dirac
%\newcommand*{\eps}{\varepsilonup}%Levi-Civita, Heaviside
%\newcommand*{\riem}{\zetaup}%Riemann zeta
%\providecommand{\degree}{\textdegree}% degree
%\newcommand*{\celsius}{\textcelsius}% degree Celsius
%\newcommand*{\micro}{\textmu}% degree Celsius
\newcommand*{\I}{\mathrm{i}}%imaginary unit
\newcommand*{\e}{\mathrm{e}}%Neper
\newcommand*{\di}{\mathrm{d}}%differential
%\newcommand*{\Di}{\mathrm{D}}%capital differential
%\newcommand*{\planckc}{\hslash}
%\newcommand*{\avogn}{N_{\textrm{A}}}
%\newcommand*{\NN}{\bm{\mathrm{N}}}
%\newcommand*{\ZZ}{\bm{\mathrm{Z}}}
%\newcommand*{\QQ}{\bm{\mathrm{Q}}}
\newcommand*{\RR}{\bm{\mathrm{R}}}
%\newcommand*{\CC}{\bm{\mathrm{C}}}
%\newcommand*{\nabl}{\bm{\nabla}}%nabla
%\DeclareMathOperator{\lb}{lb}%base 2 log
%\DeclareMathOperator{\tr}{tr}%trace
%\DeclareMathOperator{\card}{card}%cardinality
%\DeclareMathOperator{\im}{Im}%im part
%\DeclareMathOperator{\re}{Re}%re part
%\DeclareMathOperator{\sgn}{sgn}%signum
%\DeclareMathOperator{\ent}{ent}%integer less or equal to
%\DeclareMathOperator{\Ord}{O}%same order as
%\DeclareMathOperator{\ord}{o}%lower order than
%\newcommand*{\incr}{\triangle}%finite increment
\newcommand*{\defd}{\coloneqq}
\newcommand*{\defs}{\eqqcolon}
%\newcommand*{\Land}{\bigwedge}
%\newcommand*{\Lor}{\bigvee}
%\newcommand*{\lland}{\DOTSB\;\land\;}
%\newcommand*{\llor}{\DOTSB\;\lor\;}
%\newcommand*{\limplies}{\mathbin{\Rightarrow}}%implies
%\newcommand*{\suchthat}{\mid}%{\mathpunct{|}}%such that (eg in sets)
%\newcommand*{\with}{\colon}%with (list of indices)
%\newcommand*{\mul}{\times}%multiplication
%\newcommand*{\inn}{\cdot}%inner product
%\newcommand*{\dotv}{\mathord{\,\cdot\,}}%variable place
%\newcommand*{\comp}{\circ}%composition of functions
%\newcommand*{\con}{\mathbin{:}}%scal prod of tensors
%\newcommand*{\equi}{\sim}%equivalent to 
\renewcommand*{\asymp}{\simeq}%equivalent to 
%\newcommand*{\corr}{\mathrel{\hat{=}}}%corresponds to
%\providecommand{\varparallel}{\ensuremath{\mathbin{/\mkern-7mu/}}}%parallel (tentative symbol)
\renewcommand*{\le}{\leqslant}%less or equal
\renewcommand*{\ge}{\geqslant}%greater or equal
\DeclarePairedDelimiter\clcl{[}{]}
%\DeclarePairedDelimiter\clop{[}{[}
%\DeclarePairedDelimiter\opcl{]}{]}
%\DeclarePairedDelimiter\opop{]}{[}
\DeclarePairedDelimiter\abs{\lvert}{\rvert}
%\DeclarePairedDelimiter\norm{\lVert}{\rVert}
\DeclarePairedDelimiter\set{\{}{\}}
%\DeclareMathOperator{\pr}{P}%probability
\newcommand*{\pf}{\mathrm{p}}%probability
\newcommand*{\p}{\mathrm{P}}%probability
%\newcommand*{\E}{\mathrm{E}}
%\renewcommand*{\|}{\nonscript\,\vert\nonscript\;\mathopen{}}
\renewcommand*{\|}[1][]{\nonscript\,#1\vert\nonscript\;\mathopen{}}
%\DeclarePairedDelimiterX{\cond}[2]{(}{)}{#1\nonscript\,\delimsize\vert\nonscript\;\mathopen{}#2}
%\DeclarePairedDelimiterX{\condt}[2]{[}{]}{#1\nonscript\,\delimsize\vert\nonscript\;\mathopen{}#2}
%\DeclarePairedDelimiterX{\conds}[2]{\{}{\}}{#1\nonscript\,\delimsize\vert\nonscript\;\mathopen{}#2}
%\newcommand*{\+}{\lor}
%\renewcommand{\*}{\land}
%% symbol = for equality statements within probabilities
%% from https://tex.stackexchange.com/a/484142/97039
\newcommand*{\eq}{\mathrel{\!=\!}}
\let\texteq\=
\renewcommand*{\=}{\TextOrMath\texteq\eq}
%%
\newcommand*{\sect}{\S}% Sect.~
\newcommand*{\sects}{\S\S}% Sect.~
\newcommand*{\chap}{ch.}%
\newcommand*{\chaps}{chs}%
\newcommand*{\bref}{ref.}%
\newcommand*{\brefs}{refs}%
%\newcommand*{\fn}{fn}%
\newcommand*{\eqn}{eq.}%
\newcommand*{\eqns}{eqs}%
\newcommand*{\fig}{fig.}%
\newcommand*{\figs}{figs}%
\newcommand*{\vs}{{vs}}
\newcommand*{\eg}{{e.g.}}
\newcommand*{\etc}{{etc.}}
\newcommand*{\ie}{{i.e.}}
%\newcommand*{\ca}{{c.}}
\newcommand*{\foll}{{ff.}}
%\newcommand*{\viz}{{viz}}
\newcommand*{\cf}{{cf.}}
%\newcommand*{\Cf}{{Cf.}}
%\newcommand*{\vd}{{v.}}
\newcommand*{\etal}{{et al.}}
%\newcommand*{\etsim}{{et sim.}}
%\newcommand*{\ibid}{{ibid.}}
%\newcommand*{\sic}{{sic}}
%\newcommand*{\id}{\mathte{I}}%id matrix
%\newcommand*{\nbd}{\nobreakdash}%
%\newcommand*{\bd}{\hspace{0pt}}%
%\def\hy{-\penalty0\hskip0pt\relax}
%\newcommand*{\labelbis}[1]{\tag*{(\ref{#1})$_\text{r}$}}
%\newcommand*{\mathbox}[2][.8]{\parbox[t]{#1\columnwidth}{#2}}
%\newcommand*{\zerob}[1]{\makebox[0pt][l]{#1}}
\newcommand*{\tprod}{\mathop{\textstyle\prod}\nolimits}
\newcommand*{\tsum}{\mathop{\textstyle\sum}\nolimits}
%\newcommand*{\tint}{\begingroup\textstyle\int\endgroup\nolimits}
%\newcommand*{\tland}{\mathop{\textstyle\bigwedge}\nolimits}
%\newcommand*{\tlor}{\mathop{\textstyle\bigvee}\nolimits}
%\newcommand*{\sprod}{\mathop{\textstyle\prod}}
%\newcommand*{\ssum}{\mathop{\textstyle\sum}}
%\newcommand*{\sint}{\begingroup\textstyle\int\endgroup}
%\newcommand*{\sland}{\mathop{\textstyle\bigwedge}}
%\newcommand*{\slor}{\mathop{\textstyle\bigvee}}
%\newcommand*{\T}{^\intercal}%transpose
%%\newcommand*{\QEM}%{\textnormal{$\Box$}}%{\ding{167}}
%\newcommand*{\qem}{\leavevmode\unskip\penalty9999 \hbox{}\nobreak\hfill
%\quad\hbox{\QEM}}

%%%%%%%%%%%%%%%%%%%%%%%%%%%%%%%%%%%%%%%%%%%%%%%%%%%%%%%%%%%%%%%%%%%%%%%%%%%%
%%% Custom macros for this file @@@
%%%%%%%%%%%%%%%%%%%%%%%%%%%%%%%%%%%%%%%%%%%%%%%%%%%%%%%%%%%%%%%%%%%%%%%%%%%%
\definecolor{notecolour}{RGB}{68,170,153}
% \newcommand*{\puzzle}{{\fontencoding{U}\fontfamily{fontawesometwo}\selectfont\symbol{225}}}
% %\newcommand*{\puzzle}{\maltese}
% \newcommand{\mynote}[1]{ {\color{notecolour}\puzzle\ #1}}
\newcommand*{\wrench}{{\fontencoding{U}\fontfamily{fontawesomethree}\selectfont\symbol{114}}}
\newcommand{\mynote}[1]{ {\color{notecolour}\wrench\ #1}}

\newcommand*{\widebar}[1]{{\mkern1.5mu\skew{2}\overline{\mkern-1.5mu#1\mkern-1.5mu}\mkern 1.5mu}}

%\DeclareMathOperator*{\argsup}{arg\,sup}
\newcommand*{\dob}{degree of belief}
\newcommand*{\dobs}{degrees of belief}
\newcommand*{\bit}{\mathrm{bit}}

\newcommand*{\yffa}[1]{f_{#1\|\alphaup}}
\newcommand*{\yffb}[1]{f_{#1\|\betaup}}
\newcommand*{\yfa}{\bm{f}_{\!\|\alphaup}}
\newcommand*{\yfb}{\bm{f}_{\!\|\betaup}}
\newcommand*{\yFFa}[1]{F_{#1\|\alphaup}}
\newcommand*{\yFFb}[1]{F_{#1\|\betaup}}
\newcommand*{\yFa}{\bm{F}_{\!\|\alphaup}}
\newcommand*{\yFb}{\bm{F}_{\!\|\betaup}}
\newcommand*{\vsum}{\bm{\varSigma}}
\newcommand*{\hh}{\textrm{H}}
\newcommand*{\dd}{\textrm{D}}
\newcommand*{\ii}{I}%_{\textrm{h}}}
\newcommand*{\yIv}{\Hat{I}}%_{\textrm{h}}}
%\DeclareMathOperator*{\lb}{lb}


%%% Custom macros end @@@

%%%%%%%%%%%%%%%%%%%%%%%%%%%%%%%%%%%%%%%%%%%%%%%%%%%%%%%%%%%%%%%%%%%%%%%%%%%%
%%% Beginning of document
%%%%%%%%%%%%%%%%%%%%%%%%%%%%%%%%%%%%%%%%%%%%%%%%%%%%%%%%%%%%%%%%%%%%%%%%%%%%
%\firmlists
\begin{document}
\captiondelim{\quad}\captionnamefont{\footnotesize}\captiontitlefont{\footnotesize}
\selectlanguage{british}\frenchspacing
\maketitle

%%%%%%%%%%%%%%%%%%%%%%%%%%%%%%%%%%%%%%%%%%%%%%%%%%%%%%%%%%%%%%%%%%%%%%%%%%%%
%%% Abstract
%%%%%%%%%%%%%%%%%%%%%%%%%%%%%%%%%%%%%%%%%%%%%%%%%%%%%%%%%%%%%%%%%%%%%%%%%%%%
\abstractrunin
\abslabeldelim{}
\renewcommand*{\abstractname}{}
\setlength{\absleftindent}{0pt}
\setlength{\absrightindent}{0pt}
\setlength{\abstitleskip}{-\absparindent}
\begin{abstract}\labelsep 0pt%
  \noindent A reasoned analysis of inference for long-run mutual
  information between stimuli and responses from small samples is given.
  The use of estimators, biased or not, is found to be inadequate for the
  small-sample case. Moreover, any inference or formula for bias is found
  to heavily depend on the specific peculiarities of the problem -- the
  specific kind of stimuli and responses, brain region, behavioural and
  environmental conditions, and so on -- making any one-fits-all formula
  universally poor.
\end{abstract}
\selectlanguage{british}\frenchspacing

%%%%%%%%%%%%%%%%%%%%%%%%%%%%%%%%%%%%%%%%%%%%%%%%%%%%%%%%%%%%%%%%%%%%%%%%%%%%
%%% Epigraph
%%%%%%%%%%%%%%%%%%%%%%%%%%%%%%%%%%%%%%%%%%%%%%%%%%%%%%%%%%%%%%%%%%%%%%%%%%%%
% \asudedication{\small ***}
% \vspace{\bigskipamount}
\setlength{\epigraphwidth}{.56\columnwidth}
%\epigraphposition{flushright}
%\epigraphtextposition{flushright}
%\epigraphsourceposition{flushright}
\epigraphfontsize{\footnotesize}
\setlength{\epigraphrule}{0pt}
%\setlength{\beforeepigraphskip}{0pt}
%\setlength{\afterepigraphskip}{0pt}
% \epigraph{\emph{% In allowing us to focus on one aspect of a concept (e.g., the battling aspects of arguing), a metaphorical concept can keep us from focusing on other aspects of the concept that are inconsistent with that metaphor. For example, i
% \textins*{I}n the midst of a heated argument, when we are intent on attacking our opponent's position and defending our own, we may lose sight of the cooperative aspects of arguing. Someone who is arguing with you can be viewed as giving you his time, a valuable commodity, in an effort at mutual understanding.% But when we are preoccupied with the battle aspects, we often lose sight of the cooperative aspects.
% }}{\parencite[\chap~4 p.~10]{lakoffetal1980}}


%%%%%%%%%%%%%%%%%%%%%%%%%%%%%%%%%%%%%%%%%%%%%%%%%%%%%%%%%%%%%%%%%%%%%%%%%%%%
%%% BEGINNING OF MAIN TEXT
%%%%%%%%%%%%%%%%%%%%%%%%%%%%%%%%%%%%%%%%%%%%%%%%%%%%%%%%%%%%%%%%%%%%%%%%%%%%

\mynote{draft comments can be introduced with the macro \texttt{\textbackslash mynote\{\}}}

%\input{Introduction.tex}

%\input{L-Samples-are-misleading.tex}


%\input{S-Long-run-freqs.tex}

\section{Likelihood of long-run frequencies}\label{sec:likelihood} 

%Claudia Examples with equally likely long-run freqs – we can have very different
%longrun MI leading to same sample MI. Introduced by a discrete example.
%Multinomial probabilities

In the previous section we showed that mutual information between the stimulus $s$ and the response $r$ computed directly from the sample may significantly vary across samples from the same population. In reality we don't typically have access to multiple samples, but we have to rely on one single sample to make an estimate of the population mutual information. As explained in section ... the population mutual information is a function of the population frequencies, or long-run frequencies, which are unknown. Yet we can use our data to inform us on the long-run frequencies that may have generated our sample and use them to estimate the population mutual information. How do we appraise candidate long-run frequencies for our data? Given the candidate long-run frequencies $\mathbf{f_s}=\{f_s(r)\}_{s,r}$ and a sample $\hat{s}$, the likelihood of these long-run frequencies $P(\hat{s}\vert \mathbf{f_s})$ is the probability that our sample was generated from a population with such long-run frequencies. For the dataset of spike-count responses and north/south head direction stimulus introduced in section ... the likelihood of long-run frequencies $\mathbf{f_s}$ is: 


\begin{equation}
P(\hat{s}\vert \mathbf{f_s})=\prod_{s,r} f_s(r)^{N\hat{f}_s(r)}
%\frac{N!}{\prod_{s,r} \hat{f}_s(r)!}\prod_{s,r} f_s(r)^{N\hat{f}_s(r)}
\label{eq:categorical_likelihood}
\end{equation}

where $\hat{f}_s(r)$ are the sample frequencies shown in Fig.~\ref{fig:sample_frequencies}. 

It can be easily seen that the long-run frequencies that maximize the likelihood in Eq. \ref{eq:categorical_likelihood} are the sample frequencies $f_s(r)=\hat{f}_s(r)$. Thus, in order to estimate the population mutual information between head direction and spike count, it might be tempting to simply compute the mutual information from the sample frequencies being the maximum-likelihood ones, while disregarding other potential long-run frequencies. Such point estimator of the population mutual information has indeed the convenient property of being asymptotically optimal, meaning that for infinite data size, the maximum likelihood estimator converges to the population one \textcolor{red}{Ref}. 

In real experiments the data sample size is always limited, which is why the maximum-likelihood estimator can be a poor estimator in practice. Several authors \textcolor{red}{Refs} have indeed pointed out that the maximum-likelihood estimate of the mutual information can be biased, because of the very nature of the mutual information being positively defined. Beyond its bias the one-shot maximum-likelihood estimation is problematic in many ways. 
Ultimately the main (and fixable!) issue is that it's limited to a single candidate long-run frequency, while either previous knowledge about the system, or the likelihood itself, might indicate that other long-run frequencies should be considered. In the next two subsections we illustrate this point by considering the three long-run frequencies in Fig. \ref{fig:highlike_long-run_freqs} for the sample $\hat{s}$, introduced in section ..., from which we want to estimate the population mutual information between head direction and spike count.

%But the maximum-likelihood mutual information is point estimator and as such it does not incorporate a measure of the uncertainty associated to such estimate. Intuitively one realizes that by choosing the maximum-likelihood frequencies and computing the mutual information from them, one disregards the actual value of the likelihood.  Such value might be instead exploited to express our degree of belief (and uncertainty) on our mutual information estimate, as inherited from the degree of belief on the long-run frequencies. We will elaborate on how in the next sections.

%In the previous section we introduced the definition of Mutual Information between stimulus and responses (some measure/parametrization of neural activity), as a function of the long-run conditional frequencies of the responses on the stimulus. Unfortunately, for any biological neural system, the long-run frequencies are unknown, while what we have at our disposal is typically a limited\footnote{limited is a vague word. In this work we consider a sample limited if the number of datapoints is of the same order of magnitude of the possible combinations of stimulus and response value.} sample drawn from these unknown long-run frequencies. In order to compute the mutual information between stimulus and response, we hence want to use our data to make a guess of the long-run frequencies. How can we tell which long-run frequencies would most likely have generated our data? Let's think at the problem in reverse: if we knew the long-run frequencies then the likelihood $P(\hat{s}\vert \mathbf{f_s})$ would tell us how likely is our sample. It follows that for in the case of unknown long-run frequencies, the likelihood of \textit{candidate} long-run frequencies for our data can instruct us on how likely they might have generated the sample. How to choose the candidate long-run frequencies though? And, should be the likelihood the only metric for appraising long-run frequencies? One might be tempted to chose as a candidate only the long-run frequencies that maximize the likelihood for our data, namely the sample frequencies. Let's see why this problematic for a limited sample:

\subsection*{High-likelihood long-run frequencies beyond the maximum-likelihood ones}
By choosing the maximum likelihood estimator of the population frequencies and computing the mutual information just from it, one disregards completely other long-run frequencies which might have only a slightly smaller likelihood. 

As an example consider the three candidate long run frequencies for the sample $\hat{s}$ in sec... displayed in Fig.\ref{fig:highlike_long-run_freqs}. They look pretty similar and all have log-likelihoods which deviates less then $10\%$ from the maximum of the log-likelihood. Although these long-run frequencies might have generated our data with similar probability (likelihood), they yield rather different values of mutual information between spike count and head direction ( MIs up to $50\%$ apart). The examples in Fig.\ref{fig:highlike_long-run_freqs} therefore suggest that the maximum-likelihood mutual information (bottom of Fig.\ref{fig:highlike_long-run_freqs}) alone  might be poorly representative of the spectrum of mutual information values corresponding to high-likelihood long-run frequencies. As a consequence, we argue that a logical approach to mutual information estimation must encompass a range of long-run frequencies, while taking into account their likelihood. 


\subsection*{Maximum-likelihhod estimates disregard prior information} 

By choosing the maximum-likelihood frequencies as the only candidate long-run frequencies for our data, one approaches the sample in a completely agnostic fashion. The assumption made by maximum-likelihood estimation is indeed that the researcher doesn't have any kind of pre-sample knowledge about the biologically plausible candidate long-run frequencies, or in other words, that the system of interest could in principle attain any population frequency with equal probability, until the sample is collected. 
Such an assumption is typically wrong, as:

\begin{enumerate}

\item \textbf{the very same neural system might have been probed before}, providing us evidence in favor of some candidate long-run frequencies over others. Suppose for example that, before collecting our sample $\hat{s}$ in fig ... , we had recorded the activity of the same neuron, while the animal was exploring the same environment. Suppose that from this recording we estimated a population mutual information between spike count and head direction of $0.1$ (an its associated uncertainty). When we now are about to collect the new sample $\hat{s}$ in fig ..., we don't just want to disregard this previous information on the system and regard every frequency as a priori equally good candidate for generating the new sample. We would rather consider the long-run frequency at the top of Fig. \ref{fig:highlike_long-run_freqs} as a better candidate than the one at the bottom of Fig. \ref{fig:highlike_long-run_freqs}, since the latter corresponds to a much higher mutual information. 

\item \textbf{the neural system of interest might have been investigated before and its results reported in the literature}. For our example of spike count vs head direction, we might know that the region we are recording from is characterized by bursty neurons with low, but different from zero, baseline firing rate. A-priori we would therefore assign a larger probability to the long-run frequency in the middle of Fig. \ref{fig:highlike_long-run_freqs} than to those at the bottom of Fig. \ref{fig:highlike_long-run_freqs}, for which the neuron is always silent in one of the two head-direction conditions.  

\item \textbf{biological constraints on the long-run frequencies for the system under investigation might be well established}. Assigning equal probability to every long-run frequency means that before conducting our experiment we regard long-run frequencies which are not attainable by the system of interest on equal footing with those actually attainable. In our example, where the response variable is the spike count, we know that a long-run frequency corresponding to an average firing rate larger than $500$Hz violates physiological constraints on the neural activity. Intuitively we want to assign such long-run frequencies very low, if not zero, a priori probability.

\end{enumerate}

Once the sample $\hat{s}$ is collected  and attributes maximum likelihood to the long-run frequency at the bottom of Fig. \ref{fig:highlike_long-run_freqs}, we intuitively want to combine this new information on long-run frequencies provided by the likelihood, with the old (prior) piece of information about the system. Disregarding prior information would simply be poor practice.

\vspace{1cm}
In this section we explained why considering only one long-run frequency, even if it's the log-likelihood one with its nice asymptotic convergence to the truth, may be reductive for the purpose of estimating the mutual information.
So if a single estimator of the mutual information based on solely the likelihood of the long-run frequencies is not a good estimator of the mutual info: how to construct a better one by encompassing a range of candidate long-run frequencies and incorporating prior knowledge on the system? We answer to this question in section ..., whereas in section ... we elaborate on how to formally express prior knowledge about the system.

\begin{figure}
\centering
\includegraphics[scale=0.5]{HighLike_Plots1.pdf}\\ 
\includegraphics[scale=0.5]{HighLike_Plots0.pdf}\\
\includegraphics[scale=0.5]{HighLike_Plots_MaxLike.pdf} 
\label{fig:highlike_long-run_freqs}
\caption{High likelihood long-run frequencies for the data in sec. \ref{example1}. Examples of high likelihood long-run frequencies corresponding to low MI (top), medium MI (center), high MI and maximum likelihood (bottom). }
\end{figure}



\section{Priors}\label{sec:priors} 

In the previous section we calculated the likelihood of three candidate long-run frequencies. The examples presented in figure~\ref{fig:highlike_long-run_freqs} exhibit similar and high likelihood values, which suggests that the data could have well been generated from those three distributions of long-run frequencies. However, we observed that those distributions lead to different values of long-run mutual information (MI). This is a direct consequence of having limited samples; the more samples are available, the more the sample frequencies will converge to the long-run frequencies and thus the sample MI will converge to the long-run MI (see section ??). When the data is scarce, however, we can tackle the uncertainty in the inference of the long-run frequencies by making use of all relevant information about the system of study available at our disposal. In the context of computing mutual information, this translates into assigning weights to the candidate long-run frequencies the data could have been sampled from. Those weights will express our prior knowledge of the data and thus will depend on characteristics of the study: for example the employed recording technique, the brain area under study, the stimulus applied, the repertoire of possible neuronal responses, etc. Based on such weights we define the \textit{superdistribution} as the distribution of all candidate long-run frequencies. Note that this distribution lives in the space of candidate long-run frequencies. 
\mynote{Don't know if I should add an equation here to illustrate the superdistribution}

In order to illustrate what the superdistribution is, let us consider a discrete space of long-run frequencies that encompasses one, two or three spikes. We can conceptualize these distributions as lying on the surface of a triangle (see figure~\ref{fig:schematic_superdistribution}), where each dot illustrates one candidate long-run frequency, and the vertexes corresponds to firing \textit{only} one, two or three spikes. One possible superdistribution would be the one that favours the vertexes (figure~\ref{fig:schematic_superdistribution}B). This, however, is not a biologically realistic assumption as we know cells fire stochastically and do not always fire exactly the same number of spikes per bin. Based on the assumption that the cell under study has a low firing rate, and most likely only fires one or two spikes per bin, we could instead choose a superdistribution that assigns larger weights to those histograms (figure~\ref{fig:schematic_superdistribution}C). 



\begin{figure}
	\centering
	\includegraphics[scale=0.35]{superdistribution.jpg}%\\
	\label{fig:schematic_superdistribution}
	\caption{(DRAFT OF FIGURE) A: Schematic of the candidate long-run frequencies space. Each histogram corresponds to one candidate long-run frequency. B-C: Candidate long-run frequencies space (top) and superdistribution (bottom) for two different superdistributions. The red dots indicate the weights assigned to each long-run frequency. (check with Luca)}
\end{figure}


In the example of the NS cell figure ?? we presented three candidate long-run frequencies among all possible candidate distributions (figure~\ref{fig:highlike_long-run_freqs}). One possible superdistribution is the one that assigns the same weight to all candidate long-run frequencies, including the three examples. While this is a valid superdistribution (all superdistributions are) it is not biologically plausible, as it assigns equal weights to distributions that favour spiking at very high rates as well as non-spiking at all. Another possible superdistribution is the one that assigns weights different from zero to the distributions present in figure~\ref{fig:highlike_long-run_freqs}, and weights equal to zero to the rest. This could represent an improvement as the distributions with non-biologically plausible firing would get weights equal to zero. As for the three long-run frequencies shown in figure~\ref{fig:highlike_long-run_freqs}, we could assign their weights based on what we know a priori about this neuron. For example, this cell has a mean firing rate of 18.8 Hz, which favours histograms A C. If in addition we know that the cell tends to fire in bursts of two spikes, then the superdistribution should favour histogram A.  

Now let us turn to more general examples that illustrate superdistrubutions over a continuum of candidate long-run frequencies. From now on we will use the terms \textit{prior} and \textit{superdistribution} interchangeably. 

\begin{enumerate}[wide,label=(\roman*)] 
	
	\item {Uniform superdistribution over frequencies:} When there is no information available about the system under study the only choice left is to use an uninformative prior, which in a discrete case consists of a uniform distribution over all candidate long-run frequencies. However, in a continuous space, this doesn't hold anymore because this will depend on the parametrization of the space. We could, for example, choose a superdistribution that gives equal weights to equal intervals of conditional frequencies $f(r \|s)$. That is, the same weight to each hypercube $\prod_{r,s}\clcl{f(r\|s), f(r \|s) + \Delta}$, for fixed $\Delta$, for all values of $f(r \|s)$. Such a superdistribution is proportional to $\prod_{rs}\di f(r \|s)$.
		
	\item  {Uniform superdistribution over sequences:} Given some conditional frequencies $f(r \|s)$, a \textit{sequence} is defined as a set of n responses for each stimulus, with the responses appearing with frequencies $f(r \|s)$. Therefore, instead of using equal weights over the frequencies, we could use equal weights over the sequences. In this case the weights given to equal intervals in the frequency space will not be uniform, because some frequencies are realized by more sequences than others. Then we would obtain a superdistribution proportional to $\prod_{s} M\{f(r \|s)\}\,\di f(r \|s)$, where $M$ is a multinomial coefficient. \mynote{from Luca: will write the exact formula}
	
	\item {Non-uniform superdistribution over sequences:} Choosing a uniform prior over the sequences could be debatable from a biological point of
	view. If the responses for instance represent the firing rate of a population of cells, low responses should generally be expected more often than very high responses. Thus, more weight should be given to sequences in which low responses occur more frequently than high responses. This would
	lead to yet another superdistribution on the space of frequencies. \mynote{from Luca: will write the exact formula}
	
	\item {Not factorizable distribution over stimuli:} Should the superdistribution be factorizable over the frequency distributions? For example, for the two stimuli considered in the example above? Owing to biological constraints some similarity across the distributions should be expected. Thus such factorizability might be not be a sensible assumption.
	
\end{enumerate}

Different superdistributions reflect different assumptions about the data. But to which extent does the choice of superdistribution affect long-run MI? We explore this by comparing the values of long-run MI obtained with the four superdistributions listed above. We proceed as follows: From a given superdistribution we sample a pair of long-run response-frequency distributions. From this pair calculate the long-run mutual information. We then sample 20+20 responses from the pair, and calculate the sample mutual information obtained from such sample. We repeat this process ?? times. Figure~\ref{fig:superdistributions}  shows a scatter plot that tells us how often we should observe every pair of \[(\text{\small long-run mutual info},
\text{\small sample mutual info})\] under the assumption of the given superdistribution, for the four superdistributions described above. 

\begin{figure}[p]%{l}{0.5\linewidth} % with wrapfigure
	\centering\includegraphics[width=0.49\linewidth]{scripts/scatter_unif10.png}%
	\hspace*{\stretch{1}}%  
	\includegraphics[width=0.49\linewidth]{scripts/scatter_peakcentre10.png}%
	\\[10\jot]%
	\includegraphics[width=0.49\linewidth]{scripts/scatter_biolunif.png}%
	\hspace*{\stretch{1}}%  
	\includegraphics[width=0.49\linewidth]{scripts/scatter_Hunif_peak.png}%
	\\%
	\caption{Top left: uniform over frequencies. Top right: more uniform over
		sequences. Bottom left: low responses preferred. Bottom right: not
		factorizable over stimuli (positive correlation;
		hierarchic)}\label{fig:superdistributions}
\end{figure}% finite_sampling_bullshit.R


\mynote{Luca wrote this: Consequently, any
	inferences of long-run from sample and any quantifications of
	\enquote{bias} heavily depend on the assumed superdistribution. -- I would remove it from here and place it in the bias section}

The four examples of figure~\ref{fig:superdistributions} show that the joint distribution of long-run \amp\ sample mutual informations can be wildly different depending on the assumed superdistribution. Therefore, choosing a \enquote{default} superdistribution to be universally used for this kind of inference \citep[\cf][]{nemenmanetal2004} is not a sensible option, as any one-fits-all choice would simply fit every concrete case extremely poorly. At the same time, not choosing a superdistribution is impossible, as any proposed algorithm or formula to infer the long-run MI from the sample one is explicitly or implicitly choosing a superdistribution. Hiding such a choice will just lead to the same poor inferences as a default choice. We are then left with the need to choose a superdistribution as best as possible from considerations of the specific study. Any such choice, even if based on a very cursory analysis, will always be better than any default choice that completely disregards the specific case.



%\section{Posterior of long-run frequencies}\label{sec:pos_long-run}

In section \ref{sec:likelihood} and Example \ref{example1_con1} we argued that for a limited sample estimating the MI by maximum likelihood might be a poor choice. As the mutual information, defined in ..., is a smooth function of the long-run frequency, the problem should be traced back to the estimation of the long-run frequencies from the sample. The issue of estimating the long-run frequencies can be approached logically as follows: 

\begin{enumerate}

\item \textbf{before conducting the experiment / looking at the data} find a set (space) of candidate long-run frequencies $\{\mathbf{f_s}^{(c)}\}_{c \in C}$ and attribute a probability $P(\mathbf{f_s}^{(c)})$ to each of these candidates. Set and probabilities must be chosen based of prior knowledge about the functional and physiological properties of the neural system of interest (e.g. for a single neuron average firing rate within the brain region and tuning to stimulus, if known). 

\item \textbf{after conducting the experiment / looking at the data} update the probability of the candidate frequencies by means of the data $\hat{s}$ via the likelihood:

\begin{equation}
P(\mathbf{f_s}^{(c)}\vert\hat{s})=\frac{P(\hat{s}\vert \mathbf{f_s}^{(c)})*P(\mathbf{f_s}^{(c)})}{P(\hat{s})}
\label{eq:posteriorBayes}
\end{equation}
where $P(\hat{s})=\sum_c P(\hat{s}\vert \mathbf{f_s}^{(c)})*P(\mathbf{f_s}^{(c)})$ is a normalization factor.

\end{enumerate} 

Equation \eqref{eq:posteriorBayes} in probability theory is known under the name of Bayes theorem and $P(\mathbf{f_s}^{(c)}\vert\hat{s})$ expresses our belief into the long-run frequencies $\mathbf{f_s}^{(c)}$ after having seen the data, once we start from $P(\mathbf{f_s}^{(c)})$. 

Explain how this approach resolves all issues with maximum likelihood at least conceptually.

%\input{L-Number-of-samples.tex}% based on larger samples from the data --
                               % see data sent by Soledad

%\input{C-Number-of-samples.tex}

%\input{S-Point-estimators.tex}

%\input{Bias.tex}

%\input{Discussion.tex}

%Here it's my comment!


%%%%%%%%%%%%%%%%%%%%%%%%%%%%%%%%%%%%%%%%%%%%%%%%%%%%%%%%%%%%%%%%%%%%%%%%%%%%
%%% Acknowledgements
%%%%%%%%%%%%%%%%%%%%%%%%%%%%%%%%%%%%%%%%%%%%%%%%%%%%%%%%%%%%%%%%%%%%%%%%%%%% 
\iffalse
\begin{acknowledgements}
  \ldots to Mari \amp\ Miri for continuous encouragement and affection, and
  to Buster Keaton and Saitama for filling life with awe and inspiration.
  To the developers and maintainers of \LaTeX, Emacs, AUC\TeX, Open Science
  Framework, R, Python, Inkscape, Sci-Hub for making a free and impartial
  scientific exchange possible.
\mbox{}\hfill\autanet
\end{acknowledgements}
\fi

\bigskip

%%%%%%%%%%%%%%%%%%%%%%%%%%%%%%%%%%%%%%%%%%%%%%%%%%%%%%%%%%%%%%%%%%%%%%%%%%%%
%%% Bibliography
%%%%%%%%%%%%%%%%%%%%%%%%%%%%%%%%%%%%%%%%%%%%%%%%%%%%%%%%%%%%%%%%%%%%%%%%%%%% 
\renewcommand*{\finalnamedelim}{\addcomma\space}
\defbibnote{prenote}{{\footnotesize (\enquote{de $X$} is listed under D,
    \enquote{van $X$} under V, and so on, regardless of national
    conventions.)\par}}

\printbibliography[prenote=prenote%,postnote=postnote
]

\end{document}

%%%%%%%%%%%%%%%%%%%%%%%%%%%%%%%%%%%%%%%%%%%%%%%%%%%%%%%%%%%%%%%%%%%%%%%%%%%%
%%% Cut text (won't be compiled)
%%%%%%%%%%%%%%%%%%%%%%%%%%%%%%%%%%%%%%%%%%%%%%%%%%%%%%%%%%%%%%%%%%%%%%%%%%%% 


%%% Local Variables: 
%%% mode: LaTeX
%%% TeX-PDF-mode: t
%%% TeX-master: t
%%% End: 
